%%%%%%%%%%%%%%%%%%%%%%%%%%%%%%%%%%%%%%%%%%%%%%%%%%%%%%%%%%%%%%%%%%%%%%%
% Document class
%%%%%%%%%%%%%%%%%%%%%%%%%%%%%%%%%%%%%%%%%%%%%%%%%%%%%%%%%%%%%%%%%%%%%%%
\documentclass[ 
	openright,          % Kapitel beginnen auf einer rechten Seite
	listof=totoc,       % Verzeichnisse im Inhaltsverzeichnis
	bibliography=totoc, % Literaturverzeichnis im Inhaltsverzeichnis
	parskip=half,       % Absätze durch einen vergrößerten Zeilenabstand getrennt
	numbers=noendperiod,
	twoside=false,
	cleardoublepage=plain,
  	%draft              % Entwurfsversion
  	12pt
]{scrbook}             % Dokumentenklasse: KOMA-Script Report

%%%%%%%%%%%%%%%%%%%%%%%%%%%%%%%%%%%%%%%%%%%%%%%%%%%%%%%%%%%%%%%%%%%%%%%
% Packages
%%%%%%%%%%%%%%%%%%%%%%%%%%%%%%%%%%%%%%%%%%%%%%%%%%%%%%%%%%%%%%%%%%%%%%%
\usepackage{subfiles}

\usepackage[utf8]{inputenc}         % Input encoding (allow direct use of special characters like "ä")
\usepackage[T1]{fontenc}

\usepackage{ae}               % Fonts für pdfLaTeX, falls keine cm-super-Fonts installiert
\usepackage{microtype}        % optischer Randausgleich, falls pdflatex verwandt

\usepackage[a4paper,%
            inner=2.0cm,outer=2.0cm,bindingoffset=0.5cm,%
            top=1.5cm,bottom=1.5cm,%
            footskip=1.0cm,includeheadfoot,
			%showframe
	]{geometry}

\usepackage[english]{babel}
\selectlanguage{english}

%\usepackage[automark]{scrpage2} 	 % Schickerer Satzspiegel mit KOMA-Script
\usepackage[onehalfspacing]{setspace} % Allow the modification of the space between lines
\usepackage{booktabs}           	 % Netteres Tabellenlayout
\usepackage{multicol}                % Mehrspaltige Bereiche
\usepackage{quotchap}                % Beautiful chapter decoration
\usepackage[nohyperlinks]{acronym}  % list of acronyms and abbreviations
\usepackage[autostyle,english=american]{csquotes} % Fuer schoene Anfuehrungszeichen im Text per \enquote
\usepackage{acronym}                
\usepackage{amsmath}
\usepackage{textcomp}
\usepackage{csvsimple}				% For reading CSV files in LaTex Tables
\usepackage{pifont}					% For DING Symbols
%\usepackage{longtable}
\usepackage{pxfonts}				% Fuer Boldface in Lstlisting
\usepackage{lmodern}
\usepackage{caption}
\usepackage{ifthen}					% Fuer PGFPlots Tick Prefix Macro
\usepackage{siunitx}				% Fuer PGFPlots Tick Prefix Macro
\usepackage{fp}						% Floating Point Berechnungen


\usepackage{listings}
%\usepackage[outputdir=build, newfloat]{minted}

\usepackage{chngcntr}

\counterwithout{footnote}{chapter}	% Fussnoten durchgaengig zaehlen

\usepackage[sort&compress,numbers]{natbib}	% für Zitate


%\lstset{ 
%  backgroundcolor=\color{codeListingBg}, 
%  basicstyle=\footnotesize,        % the size of the fonts that are used for the code
%  breakatwhitespace=false,         % sets if automatic breaks should only happen at whitespace
%  breaklines=true,                 % sets automatic line breaking
%  captionpos=b,                    % sets the caption-position to bottom
%  commentstyle=\color{commentgreen},    % comment style
%  extendedchars=true,              % lets you use non-ASCII characters; for 8-bits encodings only, does not work with UTF-8
%  frame=single,	                   % adds a frame around the code
%  keepspaces=true,                 % keeps spaces in text, useful for keeping indentation of code (possibly needs columns=flexible)
%  keywordstyle=\color{blue},       % keyword style
%  numbers=left,                    % where to put the line-numbers; possible values are (none, left, right)
%  numbersep=5pt,                   % how far the line-numbers are from the code
%  numberstyle=\tiny\color{codeListingLineNumbers}, % the style that is used for the line-numbers
%  rulecolor=\color{black},         % if not set, the frame-color may be changed on line-breaks within not-black text (e.g. comments (green here))
%  showspaces=false,                % show spaces everywhere adding particular underscores; it overrides 'showstringspaces'
%  showstringspaces=false,          % underline spaces within strings only
%  showtabs=false,                  % show tabs within strings adding particular underscores
%  stepnumber=1,                    % the step between two line-numbers. If it's 1, each line will be numbered
%  tabsize=2,	                   % sets default tabsize to 2 spaces
%  title=\lstname                   % show the filename of files included with \lstinputlisting; also try caption instead of title
%}

\usepackage[pdftex]{graphicx}        % Grafiken in pdfLaTeX
\usepackage[dvipsnames]{xcolor}
\usepackage{xspace}
\usepackage{floatflt}
\usepackage{float}
%\usepackage{scrhack}
\usepackage{florianTitlepage}
\usepackage{datetime} % Fuer Datumformatierungen
\usepackage{pgfplots}
\usepackage{pgfplotstable}	% Zum Lesen in PFGPlots-Tabellen
\usepackage{tikzscale}		% Zum Umskalieren von PGFPlots
\usepackage{numprint}
\npthousandsep{\,}


%\usepgfplotslibrary{external} 
%\tikzsetexternalprefix{figures/}
%\tikzexternalize

%\newenvironment{mlisting}{\captionsetup{type=listing}}{}
%\SetupFloatingEnvironment{listing}{name=Listing}
\pgfplotsset{
	compat=newest,
   	scale only axis,
	xticklabel style={inner xsep=0pt},
	ylabel style={inner ysep=0pt},
}

%-----Für Autobuild --------
\newbool{isautobuild}
\setbool{isautobuild}{false} % Wir bauen manuell. Flag wird automatisch vom Autobuilder manipuliert.

%% Projekt information %%
\newcommand{\projname}{Student Research Project WS\,2020}
\newcommand{\titel}{Development and Evaluation of a GASPI Benchmarking Suite}
\newcommand{\untertitel}{}
\newcommand{\authorname}{Florian Rudolf Beenen (3541696)}

\day   = 5
\month = 12
\year  = 2020
\newcommand{\Datum}{\small\today}
\renewcommand{\dateseparator}{.}


\lstdefinestyle{cpp} {
		language={C++},
		backgroundcolor=\color{codeListingBg},
		basicstyle=\footnotesize,
		keywordstyle=\color{codeListingKeyword},
		commentstyle=\color{commentgreen},
		numbers=left,
 		numbersep=5pt,
  		numberstyle=\tiny\color{codeListingLineNumbers},
  		breaklines=true,
  		tabsize=2,
  		showstringspaces=false,
		morekeywords={
			shared
		}
}


\lstdefinestyle{console} {
		backgroundcolor=\color{codeListingBg},
		basicstyle=\footnotesize,
		keywordstyle=\color{codeListingKeyword},
		commentstyle=\color{commentgreen},
		numbers=left,
 		numbersep=5pt,
  		numberstyle=\tiny\color{codeListingLineNumbers},
  		breaklines=true,
  		tabsize=2,
  		showstringspaces=false,
}


%\lstdefinestyle{veri} {
%		language={Verilog},
%		backgroundcolor=\color{codeListingBg},
%		basicstyle=\footnotesize,
%		keywordstyle=\color{codeListingKeyword},
%		commentstyle=\color{commentgreen},
%		numbers=left,
% 		numbersep=5pt,
%  		numberstyle=\tiny\color{codeListingLineNumbers},
%  		breaklines=true,
%  		tabsize=2,
%  		showstringspaces=false
%}

\newcommand{\code}[1]{\texttt{#1}}
\newcommand{\grad}[1]{${#1}^\circ$}
\newcommand{\cmark}{\ding{51}}
\newcommand{\xmark}{\ding{55}}
\newcommand{\yes}{\cmark}
\newcommand{\no}{\xmark}
\newcommand{\eg}[0]{e.\,g.\ }
\newcommand{\ie}[0]{i.\,e.\ }
\newcommand{\zb}[0]{z.\,B.\ }
\newcommand{\dahe}[0]{d.\,h.\ }
\newcommand{\cross}[1]{#1\,$\times$\,#1\,}
\newcommand{\crosst}[2]{#1\,$\times$\,#2\,}
\newcommand{\chref}[1]{Chapter \ref{#1}\xspace}
\newcommand{\secref}[1]{Section \ref{#1}\xspace}
\newcommand{\gpi}[0]{\acs{GPI}-2\xspace}

\newcommand{\gaspiWriteNotify}{\code{gaspi\_\allowbreak write\_\allowbreak noti\-fy}\xspace}
\newcommand{\gaspiReadNotify}{\code{gaspi\_\allowbreak read\_\allowbreak noti\-fy}\xspace}
\newcommand{\gaspiPassiveSend}{\code{gaspi\_\allowbreak passive\_\allowbreak send}\xspace}
\newcommand{\gaspiPassiveReceive}{\code{gaspi\_\allowbreak passive\_\allowbreak receive}\xspace}
\newcommand{\gaspiAllreduce}{\code{gaspi\_\allowbreak all\-reduce}\xspace}
\newcommand{\ubenchData}{\code{gbs\_\allowbreak ubench\_\allowbreak data\_\allowbreak t}\xspace}
\newcommand{\ubenchInfo}{\code{gbs\_\allowbreak ubench\_\allowbreak bench\_\allowbreak info\_\allowbreak t}\xspace}
\newcommand{\ubenchResult}{\code{gbs\_\allowbreak ubench\_\allowbreak result\_\allowbreak t}\xspace}
\newcommand{\microbenchmarkExecute}{\code{gbs\_\allowbreak microbenchmark\_\allowbreak execute}\xspace}
\newcommand{\macroPosix}{\code{-D\_\allowbreak POSIX\_\allowbreak C\_\allowbreak SOURCE=\allowbreak199309L}\xspace}

\hyphenation{mi-cro-bench-mark}
\hyphenation{mi-cro-bench-marks}



% big O notation for runtime of algorithms
\newcommand{\bigo}[1]{$\mathcal{O}(#1)$}
\newcommand{\tblcellsplit}[1]{\begin{tabular}{@{}c@{}} #1 \end{tabular}}


\newfloat{lstfloat}{htbp}{lop}
\floatname{lstfloat}{Listing}
\def\lstfloatautorefname{Listing} % needed for hyperref/auroref


\makeatletter
\newcommand*{\textoverline}[1]{$\overline{\hbox{#1}}\m@th$}
\makeatother

% Layout
%\pagestyle{scrheadings}
%\pagestyle{empty}
\clubpenalty = 10000
\widowpenalty = 10000
\displaywidowpenalty = 10000
 
\makeatletter
\renewcommand{\fps@figure}{htbp}
\makeatother

\definecolor{customDarkblue}{rgb}{0,0,.5}
\definecolor{codeListingBg}{rgb}{0.95,0.95,0.95}
\definecolor{codeListingLineNumbers}{rgb} {0.3,0.3,0.3}
\definecolor{commentgreen}{rgb} {0.2,0.8,0.2}
\definecolor{codeListingKeyword}{rgb}{0.0,0.0,1.0}

\usepackage[pdftex,
            pdfauthor={Florian Beenen},
            pdftitle={\titel},
            pdfsubject={\projname},
            pdfproducer={pdflatex},
            pdfcreator={pdflatex}]{hyperref}

% BW-Links
\hypersetup{
	colorlinks=true,
	breaklinks=true,
	linkcolor=black,
	menucolor=black,
	urlcolor=black,
	citecolor=black,
}


% Verhindern von Doppel-Bars in Barplots in der Legende.
% https://tex.stackexchange.com/questions/224676/reason-for-multiple-bars-in-legend-entries
\pgfplotsset{
    /pgfplots/ybar legend/.style={
    /pgfplots/legend image code/.code={%
       \draw[##1,/tikz/.cd,yshift=-0.25em]
        (0cm,0cm) rectangle (3pt,0.8em);},
   },
}

% Convert PGFPlots Label Ticks to "Kilo" , "Mega" Prefix notation.
% https://tex.stackexchange.com/questions/183225/engineering-notation-scientific-as-tick-labels-with-pgfplots-maybe-via-siunit
\newcommand{\engticksilabel}[1]{
        \pgfmathparse{0}% precision
        \edef\numtoengprecision{\pgfmathresult}%
        \pgfkeys{/pgf/fpu}% else dimension too large!
        \pgfmathparse{(#1)}% % eng labels come out as logarithms
        \edef\numtoengraw{\pgfmathresult}%
        \pgfmathparse{round((\numtoengraw)*10^6)/(10^6)}%
        \pgfmathfloattosci\pgfmathresult
        \edef\numtoengroundedraw{\pgfmathresult}%
        \pgfmathparse{floor(log10(\numtoengraw)/(log10(1000.0)))}%
        \edef\numtoengexpthree{\pgfmathresult}%
        \pgfmathparse{(\numtoengraw)*(1000^((-1.0)*\numtoengexpthree))}%
        \edef\numtoenginputnormalized{\pgfmathresult}%
        \pgfmathparse{3*\numtoengexpthree}%
        \pgfmathfloattofixed\pgfmathresult%
        \edef\numtoengexpten{\pgfmathresult}%
        \pgfmathparse{round(\numtoenginputnormalized*10^\numtoengprecision)*(10^(-1*\numtoengprecision))}%
        \pgfmathfloattofixed\pgfmathresult%
        \edef\numtoenginputnormalizedrounded{\pgfmathresult}%
        \pgfmathparse{ifthenelse(or(\numtoengexpthree<-(18/3),\numtoengexpthree>(18/3)),1,0)}% too small
        \pgfmathfloattoint{\pgfmathresult}%
        \message{raw=\numtoengraw,exp3=\numtoengexpthree,normalized=\numtoenginputnormalized,normalizedrounded=\numtoenginputnormalizedrounded}%
        \ifthenelse{\equal{\pgfmathresult}{1}}%
        {%
            \edef\engtonumnumber{\num[%
                scientific-notation=engineering,%
                exponent-to-prefix=true,
                round-mode=places,%
                round-precision=3,%
                zero-decimal-to-integer,%
                group-digits=false,%
                exponent-product=\!\cdot\!,%
                ]{\numtoenginputnormalizedrounded}} 
            \edef\engtonumexptennumber{\num[%
                scientific-notation=engineering,%
                exponent-to-prefix=true,
                round-mode=places,%
                round-precision=3,%
                zero-decimal-to-integer,%
                group-digits=false,%
                exponent-product=\!\cdot\!,%
                ]{\numtoengexpten}}% 
            \def\numtoengsiprefix{{\cdot} 10^{\engtonumexptennumber}}%
        }{% else: normal number format
            \edef\engtonumnumber{\num[%
                scientific-notation=engineering,%
                exponent-to-prefix=true,
                round-mode=places,%
                round-precision=3,%
                zero-decimal-to-integer,%
                group-digits=false,%
                exponent-product=\!\cdot\!,%
                ]{\numtoenginputnormalizedrounded}}% 
        }%
        \pgfmathparse{ifthenelse(\numtoengexpthree==-(18/3),1,0)}%
        \pgfmathfloattoint{\pgfmathresult}%
        \ifthenelse{\equal{\pgfmathresult}{1}}%
        {\edef\numtoengsiprefix{\,\si{\atto}}}{}%
        \pgfmathparse{ifthenelse(\numtoengexpthree==-(15/3),1,0)}%
        \pgfmathfloattoint{\pgfmathresult}%
        \ifthenelse{\equal{\pgfmathresult}{1}}%
        {\edef\numtoengsiprefix{\,\si{\femto}}}{}%
        \pgfmathparse{ifthenelse(\numtoengexpthree==-(12/3),1,0)}%
        \pgfmathfloattoint{\pgfmathresult}%
        \ifthenelse{\equal{\pgfmathresult}{1}}%
        {\edef\numtoengsiprefix{\,\si{\pico}}}{}%
        \pgfmathparse{ifthenelse(\numtoengexpthree==-(9/3),1,0)}%
        \pgfmathfloattoint{\pgfmathresult}%
        \ifthenelse{\equal{\pgfmathresult}{1}}%
        {\edef\numtoengsiprefix{\,\si{\nano}}}{}%
        \pgfmathparse{ifthenelse(\numtoengexpthree==-(6/3),1,0)}%
        \pgfmathfloattoint{\pgfmathresult}%
        \ifthenelse{\equal{\pgfmathresult}{1}}%
        {\edef\numtoengsiprefix{\,\si{\micro}}}{}%
        \pgfmathparse{ifthenelse(\numtoengexpthree==-(3/3),1,0)}%
        \pgfmathfloattoint{\pgfmathresult}%
        \ifthenelse{\equal{\pgfmathresult}{1}}%
        {\edef\numtoengsiprefix{\,\si{\milli}}}{}%
        \pgfmathparse{ifthenelse(\numtoengexpthree==(0/3),1,0)}%
        \pgfmathfloattoint{\pgfmathresult}%
        \ifthenelse{\equal{\pgfmathresult}{1}}%
        {\edef\numtoengsiprefix{}}{}%   
        \pgfmathparse{ifthenelse(\numtoengexpthree==(3/3),1,0)}%
        \pgfmathfloattoint{\pgfmathresult}%
        \ifthenelse{\equal{\pgfmathresult}{1}}%
        {\edef\numtoengsiprefix{\,\si{\kilo}}}{}%
        \pgfmathparse{ifthenelse(\numtoengexpthree==(6/3),1,0)}%
        \pgfmathfloattoint{\pgfmathresult}%
        \ifthenelse{\equal{\pgfmathresult}{1}}%
        {\edef\numtoengsiprefix{\,\si{\mega}}}{}%
        \pgfmathparse{ifthenelse(\numtoengexpthree==(9/3),1,0)}%
        \pgfmathfloattoint{\pgfmathresult}%
        \ifthenelse{\equal{\pgfmathresult}{1}}%
        {\edef\numtoengsiprefix{\,\si{\giga}}}{}%
        \pgfmathparse{ifthenelse(\numtoengexpthree==(12/3),1,0)}%
        \pgfmathfloattoint{\pgfmathresult}%
        \ifthenelse{\equal{\pgfmathresult}{1}}%
        {\edef\numtoengsiprefix{\,\si{\tera}}}{}%
        \pgfmathparse{ifthenelse(\numtoengexpthree==(15/3),1,0)}%
        \pgfmathfloattoint{\pgfmathresult}%
        \ifthenelse{\equal{\pgfmathresult}{1}}%
        {\edef\numtoengsiprefix{\,\si{\peta}}}{}%
        \pgfmathparse{ifthenelse(\numtoengexpthree==(18/3),1,0)}%
        \pgfmathfloattoint{\pgfmathresult}%
        \ifthenelse{\equal{\pgfmathresult}{1}}%
        {\edef\numtoengsiprefix{\,\si{\exa}}}{}%
        {$\engtonumnumber\numtoengsiprefix$}%\numtoengsiprefix$}
    }

%\addto\captionsngerman{\renewcommand{\bibname}{Quellenverzeichnis}}
\addto\captionsenglish{\renewcommand{\bibname}{References}}

\bibliographystyle{thesis-florian}
\renewcommand{\arraystretch}{1.2}
